\section{Introduction}
%-------------------------------------------------------------------------------------------------------------------------------------------------------------------------------------------------------------------------------------------------------------------------------------------------------------------------------------%
\section{Terminology}
\textbf{Exposed worker/Radiation worker} Employee exposed to >1mSv/y. \\\\
\textbf{Practices} Everything that may increase one's exposure. Including storage.\\\\
\textbf{RPO/TMS} Technically competent in a specific branch of radiation protection. 9 groups: medical applications, dentistry, veterinary X-ray, nuclear fuel cycle, dispersible radiation substances, NORM, accelerators, industrial radiography, mesurement and control applications. See Annex 5.2 of Rbs for training requirements. Every company that works with ionising radiation must employ an RPO, RPO does not have to be recorded in a national register.\\\\
\textbf{RPE} Gives advice about radiation protection and is recognised by the State as an expert. They also have supervisory tasks.\\\\
\textbf{Radiation expert} RPOs and RPEs \\\\
\textbf{Undertaking} The person (or company) under whose responsibility a practice is carried out or a measure is taken.\\\\
\textbf{Exposure situations} Planned exposure situations, radiological emergency situations (immediate action), existing exposure situations.\\\\
\textbf{Radiation incident} An unintended event or situation or unintended spread of activity, in which there is or is danger of exposure to ionising radiation by public of >0.1 mSv, discharge to or into the soil, sewer, surface water or air above a specific level, or an exposure to ionising radiation of workers exceeding 2 mSv. This must be reported.\\\\
\textbf{Anticipated and non-anticipated unintended event} Whether it's in the radiation risk inventory and evaluation \\\\
\textbf{Reporting obligations} Radiation incidents + non-anticipated unintended events + anticipated unintended events if dose is higher than predicted.\\\\
%-------------------------------------------------------------------------------------------------------------------------------------------------------------------------------------------------------------------------------------------------------------------------------------------------------------------------------------%
\section{The system of radiation protection}
\textbf{Justification} The application must be justified, and no alternative non-ionising methods must exist. Pros and cons must be evaluated as individual and society (present and future). Annex 2.1 of Rbs includes justified examples. RPO and radiation worker won't notice justification, as license has also been granted.\\\\
\textbf{ALARA/optimisation}  As Low As Reasonably Achievable. Even if the dose is already low, it should be lowered if possible. It concerns everyone, present and future. Dose constraints contain ALARA (not LIMITS, these are guidances). An undertaking must establish dose constraints below the limits, unless practice results in extremely low exposure. Dose constraing of 10 $\mu$Sv as an annual dose for the public, 100x lower than the limit. ALARA is applied as a source-oriented strategy: examine whether a less risky source is possible, containing the source, shielding the source, increasing the distance between the source and the individual, and as a last step, personal measures. Reasonable = 1€ to save 10 $\mu$Sv (QALY). \\\\
\textbf{Dose limitation}  The limits are there to avoid too high a dose for an individual. They should not be used as a guide for a maximum permissible dose, they CANNOT be exceeded, and act as a safety net. See table in QRH for dose limits. The annual limit for Cat. A is almost never exceeded in the Netherlands. Dose limits do not apply to patients, or to an existing exposure situation, or to a radiological emergency, which are then reference levels. For existing exposure situations the limit is between 1 and 20 mSv/y. For emergencies, it is between 20 and 500 mSv, depending on the severity, the possibilities for taking protecive measures, and the role in the emergency response. In an emergency, anything over 100 mSv requires informed consent.\\\\
%-------------------------------------------------------------------------------------------------------------------------------------------------------------------------------------------------------------------------------------------------------------------------------------------------------------------------------------%
\section{Regulations at the workplace}
Before a new practice is carried out, a risk inventory and evaluation (Risk Assessment) must be carried out. Radiation risks during normal operations are considered, but also anticipated unintended effects (fire, loss of source, etc). The RPO will draft a concept, and the RPE will ratify it. The RA must be adapted when there's new developments, and at most must be revised every 5 years. Instruction must be given, including of all local protocols, as well as emergency protocols. Women need extra information. \\
Category A workers must receive a yearly medical visit by a registered medical radiation specialist, also before and after the employment, as well as yearly thereafter. \\
Category A and Category B workers must have a TLD badge or personal dosimeter. Category C \textbf{MUST? NOT HAVE IT}. \\
A controlled area is where it's possible to get > 6mSv/y effective dose. A supervised area it's between 1 adn 6 mSv/y effective dose. If a dose rate of > 10 $\mu$Sv/h can occur, a special sign must be placed stating so. Class B laboratory and workplace with an accelerator are controlled. Classs C and workplace with X-ray devices is supervised. Category B workers can work in controlled areas in specific circumstances. \\
The storage facility must be fire resistant for >60 mins. The dose rate outside must be <1 $\mu$Sv/h. The storage facility must be easy to decontaminate and a ventilation rate of 3x/h if dispersible sources are used.\\
TLD badges for Class A and B employees have their data recorded in NDRIS (National Dose Registration and Information System). This data must be kept until the age of 75 of the employee, OR AT LEAST 30 years after the termination of the work, whichever is longer. The badge is worn on the collar, mid-torso or waist with the label facing out; on top of an apron if it's worn. If the worker always wears an apron, the TLD company must take this into account, and a correction factor of 0.2 will apply only if the lead aprons are adequate, in the medical profession, and the voltage of X-ray apparatus is <125 kV. \\
Any information relevant for radiation protection must be kept in the Nuclear Energy Act File.
%-------------------------------------------------------------------------------------------------------------------------------------------------------------------------------------------------------------------------------------------------------------------------------------------------------------------------------------%
\section{Regulations regarding security}
The security category of a source is determined by the A (activity)/D-value. For an A/D <1, no special security measures are needed, but it's necessary to prevent loss, robbery, or dispersion. 
%-------------------------------------------------------------------------------------------------------------------------------------------------------------------------------------------------------------------------------------------------------------------------------------------------------------------------------------%
\section{Transport regulations}
Any transport should be done by an ADR-certified carrier. The transport must be notified 3 weeks in advance to the ANVS; unless done by an ADR, in which case it's a yearly notification and no 3-week notice is needed; unless for fissile material, where a license is needed. \\\\
Packages (collo/colli). For an exempted package, the total activity and activity concentration are so low, no labelling is needed. The TI level must be specified, it's 0.1x ($\mu$Sv) the radiation level 1m away from the collo. A safety label is needed:\\
\begin{enumerate}
\item I - White. Radiation level on collo surface <5 $\mu$Sv/h.
\item II - Yellow. Radiation level on collo surface <500 $\mu$Sv/h, >5 $\mu$Sv/h, must not exceed 10 $\mu$Sv/h at 1m.
\item III - Yellow. Radiation level on collo surface <2000 $\mu$Sv/h (limit MAY be exceeded in some conditions), >500 $\mu$Sv/h, must not exceed 100 $\mu$Sv/h at 1m, but higher than 10 $\mu$Sv/h at 1m.
\end{enumerate}
