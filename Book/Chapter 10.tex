\section{Introduction}
All information in this section is for the RW (Radiation Worker). The RPO must monitor that these measures are taken. Radiation risks might concur as a result of conventional hazards, called anticipated unintended events. There are also non-radiation risks. Both types of risks are included in the Radiation Risk Assessment, and the General Risk Assessment. 
%-------------------------------------------------------------------------------------------------------------------------------------------------------------------------------------------------------------------------------------------------------------------------------------------------------------------------------------%
\section{Elaboration on the legal framework}
Applications of open sources must be justified. ALARA must be applied. No dose constraints are set by the authorities (but may be set by the company), but the limits still apply.\\
There is no specific law aimed at open sources, general radiation laws apply.\\
A licence is almost always required when working with open sources, exemption levels don't matter, but clearance levels do.\\
A RA must be available before work starts, and new workers must be informed about radiation safety measures. For unusual applications, a separate RA and an internal authorisation are required.\\
Before the work: 
\begin{itemize}
    \item Verify that there is a license or internal authorization
    \item Check the maximum permitted activity with the pqr formula
    \item Make sure you are sufficiently informed about your work, risks, and mitigation measures
    \item Check which category of employee you are
    \item Consider reporting a pregnancy (NOT MANDATORY)
    \item Arrange your personal dosimeter
    \item Plan your work in accordance with source-oriented strategy
    \item Check thee facilities of your work (especially radiation detectors and fume hoods)
    \item Make sure you have enough time for your planned work
    \item If needed, practice "cold" at first
\end{itemize}
%-------------------------------------------------------------------------------------------------------------------------------------------------------------------------------------------------------------------------------------------------------------------------------------------------------------------------------------%
\section{Reducing activity}
The activity applied sometimes is high in order to obtain a quick and accurate result, but sometimes it's not needed. Counting overnight with less activity may be an option, or less accuracy. This is safer and cheaper. After preparing a work solution, the stock solution must be immediately returned to storage.
%-------------------------------------------------------------------------------------------------------------------------------------------------------------------------------------------------------------------------------------------------------------------------------------------------------------------------------------%
\section{Containment}
Measures must be taken to prevent dispersion as much as possible outside the containment. The activity should remain within the splash tray, and inside the RNL\\
An adequate splash tray must be available, and everything must be in place prior to the work. The liquid waste bottle or vessel should be in a sufficiently large drip tray. \\
Aerosol formation should be avoided. Vacuum pumps' oil should be checked if the cold trap is full or not large enough. Using dirty vials in Liquid Scintillation Counters may give wrong results. Centrifuges may be a source of contamination, as they're not always properly cleaned after leakages (warn RPO). Pipettes should be kept clean, for example by wrapping the bottom in parafilm or using filter tips. Stof in fridges or freezers should be well-closed, and parafilm should NEVER be used. \\
Doors must always be closed, as there's a negative pressure in each room. Opening doors must happen the least amount possible. The lab coat should not be worn outside the laboratory. Nothing with radioactive contact must be taken out of the laboratory without consulting the RPO. Special packaging must be used during transport. Radioactive warning signs must be employed, and radiation must be checked.
%-------------------------------------------------------------------------------------------------------------------------------------------------------------------------------------------------------------------------------------------------------------------------------------------------------------------------------------%
\section{Removal of airborne contamination}
Class B and C laboratories must have at least one properly functioning fume-hood, and the lab must also be properly ventilated (>8x/h)
\subsection{Fume hood}
The fume hood is the most used local exhaust ventilation. The air has a directional flow into the fume hood so when used correctly no air will leave the fume hood. It should not be overfilled, and the equipment should be placed as far back as possible (15-20cm to the front at least). Suction gaps should not be blocked. The window should be almost-closed as much as possible. No violent movements should be done, and distance should be kept from fume hoods in use. No agressive substances should be allowed to evaporate.
\subsection{Biosafety cabinet}
Basically, an upgraded fume hood. The air is always biologically clean. Usually it provides better protection than a normal fume hood.
\subsection{Glove box}
The most effective local exhaust ventilation. The space inside the box is completely closed off, and all interaction is done with gloves. There's negative pressure inside. They are rare, and it is tedious to work with them. They usually have an alarm button which can be operated with the knee.
\subsection{Inadequate systems}
Laminar flow cabinets are prohibited, as they have positive pressure inside, allowing radiation to escape towards the worker. Extractor hoods and point extractions don't work, other than with warm air.
%-------------------------------------------------------------------------------------------------------------------------------------------------------------------------------------------------------------------------------------------------------------------------------------------------------------------------------------%
\section{Individual protection}
Safety glasses with closed sides should be used if splashing may occur. Disposable, well-fitting gloves should be used, and they shouldn't be used when handling utensils. They should be regularly tested for contamination and replaced regularly, especially when working with VOCs, as they may affect the integrity. Gloves should not be used outside radiation areas. There may be allergy to gloves. \\
As much as possible, the single-glove method should be used.
%-------------------------------------------------------------------------------------------------------------------------------------------------------------------------------------------------------------------------------------------------------------------------------------------------------------------------------------%
\section{Contamination check and decontamination}
A regular contamination and external exposure check should be performed. If a contamination is present, the extent should be determined and labelled. Then, all material needed for contamination must be rounded up, sometimes a spill kit exists. The contaminated area should be wiped from outside to inside with damp tissues with or without detergent, but not foaming agents. \\
Radiation may be removed up to 4 Bq/cm² for $\beta$ and $\gamma$ emitters, and 0.4 Bq/cm² for $\alpha$, measured over 5 cm². Decontamination is usually continued until radiation can no longer be detected above background on the most sensitive apparatus (ALARA). If this isn't possible, RPO must be alerted. \\
If the spill is too big, it must be contained immediately, and colleagues alerted. The contaminated area must be marked, and covered with foil to prevent evaporation. The RPO must be alerted, who will supervise the decontamination.\\
When a person is contaminated or exposed, the RPO must be alerted immediately. A contaminated person must stay in place, not contaminating others. After arrival, the RPO will supervise the decontamination.
%-------------------------------------------------------------------------------------------------------------------------------------------------------------------------------------------------------------------------------------------------------------------------------------------------------------------------------------%
\section{Topics}
\subsection{Radionucleide laboratories}
Laboratories with open sources are classified as B, C, or D, increasing in strictness if the hazard is higher. The categories depend on the place where the laboratory is situated, the maximum amount of radioactivity, the construction, the layout, the processing of radioactive waste, and the level of expertise of the RPO. \\
C laboratories are the most common. All walls, floors, and bench surfaces are smooth, seamless, and easy to decontaminate. At least 1 fume hood must be present; air must be refreshed >8x/h, and there must be suitable monitors. Similar requirements exist for D laboratories, but B laboratories have much more stringent requirements. \\
To calculate the maximum permissible activity, the $A_{max}$ formula is used.