\section{Effects at the molecular and cellular level}
The DNA is the most important target for ionising radiation. The DNA can be ionised and damaged (See Chapter 3). DNA can be damaged through a single-strand, double-strand break, base or cluster (damage to closely spaced places, causing debris) damage.\\
Ionising radiation can also damage DNA and tissue through radical formation. Radicals can be formed in water: H and OH. Indirect DNA damage accounts for 2/3 of the damage, while direct damage only accounts for 1/3.\\ 
The cell can repair single-strand damage quite well, but double-strand break reparation often causes even more damage. \\
$\alpha$ radiation causes many ionisations close to each other, so double-strand breaks are common. It has the most harmful effect (x20).
%-------------------------------------------------------------------------------------------------------------------------------------------------------------------------------------------------------------------------------------------------------------------------------------------------------------------------------------%
\section{Effects in humans}
It can either be on the individual or on the offspring. On the individual:
\subsection{Stochastic/probabilitstic effects}
LNT. Random in nature, always has some probability of ocurring. They have no threshold dose, but the probability increases with increasing dose.\\
Example: cancer

\subsection{Harmful tissue reaction/deterministic/non-stochastic effects}
Takes place when enough cells in an organ have been killed. The dose needed to create an observable effect is the threshold dose. After exceeding the threshold, the severity of the effect increases with an increase of dose.\\ Example: erythrema.
%-------------------------------------------------------------------------------------------------------------------------------------------------------------------------------------------------------------------------------------------------------------------------------------------------------------------------------------%
\section{Harmful tissue reactions}
\textbf{The severity of the effect increases with the dose.} If the loss of functionality is not too serious, the recovery process will ensure the organ returns to normal (though it may take a long time and leave scar). If a dose is received over a longer period of time, more in-between recovery will take place and the long-lasting damage will be lower. It may appear immediately or after a latency period. Threshold doses are normally specified for conditions where the dose is received in a short period, with almost no recovery. \\
Threshold doses and latency periods can be seen in Table 6.1.\\\\
For a radiation worker in a diagnostic or research laboratory, the threshold cannot be exceeded, even in an incident, as the activity is too low. In medical applications, there may be damage if carelessness is present, and any threshold dose can be exceeded after an incident. For patients, the risk is greater, especially in radiotherapy. In industrial radiography, any threshold can be exceeded after an accident.
%-------------------------------------------------------------------------------------------------------------------------------------------------------------------------------------------------------------------------------------------------------------------------------------------------------------------------------------%
\section{Stochastic effects}
Data has been obtained through epidemiological studies such as: Hiroshima and Nagasaki, medical irradiation, radiation workers, Chernobyl and Fukushima, etc. Approximations* can be derived from that (*however, those approximations only apply to a select population: those healthy enough to survive the high radiation dose). This may underestimate the risk to the random group of all healthy and non-healthy people together; or it could be overestimated, and low doses can't be harmful because we're already used to them. \\
For the determination of risk, a LNT model (linear, no-threshold) is used. It means that a low additional dose also increases the risk of cancer a bit. There is plenty of debate in this topic.\\
The latency period is the period befor the cancer (late effect) becomes manifest. After the latent period, a risk follows. The latency period depends on the cancer type. Leikemia has a short latency period of 2 years, and a maximum risk of 20; lethal tumors have a latency period of 5-10 years, and risk of at least 30 years. CVD has a latent period of 40 years. \\
It's estimated that the risk of lethal cancer for $\beta$ and $\gamma$ radiation, at low dose and low dose rate is 5.5\%/sV. The risk of lethal cancer in women is 20\% higher, due to breast and thyroid cancer. Children are also more susceptible, as they have a longer life expectancy. Their risk is 2-3 times the normal public. For workers, the risk coefficient is lower, at 4.1\%/sV. This is because there's no children and usually it's healthier. However, not everyone is equally sensitive to radiation.\\ 
The probability of a non-lethal cancer is at around the same order of magnitude as the probability of lethal cancer.
%-------------------------------------------------------------------------------------------------------------------------------------------------------------------------------------------------------------------------------------------------------------------------------------------------------------------------------------%
\section{Effects on offspring}
Radiation acts as a mutagenic. The dose that the germ cells receive is called the gonad dose. The doubling dose (estimate from animal data) is the absorbed dose to the gonads that is required to produce as many hereditable mutations as those arising spontaneously in a generation (2x the mutations). No statistically significant effect, but estimated at around 1Gy at a low dose rate.
%-------------------------------------------------------------------------------------------------------------------------------------------------------------------------------------------------------------------------------------------------------------------------------------------------------------------------------------%
\section{Effects on the unborn child}
It depends on when the radiation exposure happens.\\
During the preimplantation period (first 10 days), upwards of 1 Gy will almost always lead to the death of the embryo. If they survive, it's unlikely they'll have mutations, as the cells can still differentiate.\\
During the organ formation period (10-40 days), failure or damage to cells can lead to organ malformations. The threshold is set to 100 mGy.\\
During the fetal development (the rest), the sensitivity lowers, but growth disorders and functional defects may happen. The threshold is set to 100 mGy.\\
The brain is separate. Irradiation during week 8 and week 25 leads to a decrease in IQ of 25-30 pts/Gy. In weeks 16 to 25, it's 1/4th.\\
It may be LNT, but it's effectively considered as deterministic.
%-------------------------------------------------------------------------------------------------------------------------------------------------------------------------------------------------------------------------------------------------------------------------------------------------------------------------------------%
\section{Comparison with other risks}
The risk of death can be compared to the risk of laboral deaths. The risk of radiation is much lower than the normal risk. With the legal limits, it is less risky than working with carcinogenic chemicals. The risk is much lower and can be managed.