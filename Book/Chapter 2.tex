%-------------------------------------------------------------------------------------------------------------------------------------------------------------------------------------------------------------------------------------------------------------------------------------------------------------------------------------%
\section{Definitions}
\textbf{Radiation sources}: entities that may cause exposure, including radioactive, X-ray, and accelerators. Some States call them \textbf{Sources}.\\\\
\textbf{Radioactive substances}: any substance that contains one or more radionuclide in the activity or activity concentration of which cannot be disregarded from a safety point of view.\\\\
\textbf{Radioactive sources}: radiation source incorporating radioactive material for the purpose of utilising its radioactivity. Include sealed and dispersible sources.\\\\
\textbf{Sealed sources}: the radioactive material is PERMANENTLY sealed in a capsule or incorporated in a solid form to prevent under normal use the dispersion of radiation.\\\\
\textbf{HASS/HAS-bron}: High Activity Sealed Source/Bron. Have additional regulations. \\\\
\textbf{Open/dispersible/unsealed sources}. there's a chance of dispersion.\\\\
\textbf{X-ray equipment}: equipment that can emit ionising radiation but doesn't contain any radioactive source, fissile material, or ore. When the equipment is off, no radiation is produced.\\\\
\textbf{Fissile material}: substances containing some percentage of uranium, plutonium, thorium, or other designated elements.\\\\
\textbf{Ore}: contain at least 10\% uranium or 3\% thorium, used for their fission or fertile properties. They are more regulated. Some substances, such as uranyl acetate (electron microscope stain) is considered a fissile material. Other States may use other definitions
\\\\\textbf{Sources of natural radiation}: may originate in the Earth or Space. Also known as NORM (Naturally Occurring Radioactive Material). Fewer regulations apply, unless used for their radioactive properties.
\\\\\textbf{Artificial sources}: sources with man-made radioactive substances.
%-------------------------------------------------------------------------------------------------------------------------------------------------------------------------------------------------------------------------------------------------------------------------------------------------------------------------------------%
\section{Overview of applications}
About 7500 licenses are granted in the Netherlands.\\ 
Academic hospitals (8 licenses) have the most extensive licenses, concerning X-ray equipment, some sealed sources, and most of the open sources in the Netherlands. \\
HASS are used in 4 companies that carry out non-destructive testing. \\
Less strong sealed sources are used in measurement and control technology. \\
Open sources are mostly used in radionuclide laboratories of universities or research institutions. \\
X-ray equipment is used in over 250 academic and non-academic hospitals and outpatient clinics, 4700 dental practices and 2400 veterinary practices. 
%-------------------------------------------------------------------------------------------------------------------------------------------------------------------------------------------------------------------------------------------------------------------------------------------------------------------------------------%