\section{Risk analysis}
\subsection{Why draw up a risk analysis?}
A risk analysis is drawn up to identify the riskiest parts of the job. It is mandatory by law. It must be drawn up before working with or employees being exposed to sources of ionizing radiation, prior to performing the actions. The risk analysis includes employees working with sources of ionizing radiation, other employees, visitors, and environment (site boundary). It does not include patients. It is used as an up-to-date quantification of the (possible) exposure.

\subsection{Who draws up a risk analysis?}
It is drawn up by the Radiation Protection Officer (RPO/TMS) of each laboratory. The reasercher supplies all necessary data. It's essential for the Local Internal Permit at our RNL.

\subsection{What does a risk analysis look like?}
The legal framework defines required components, such as regular exposure and potential exposure (Forseeable but Unwanted Events: such as a small spill, subjects not behaving -mouse may bite you, a sheep may vomit or poop-. It's dependent on probability, frequency, and danger). It is required to calculate the exposure for non-exposed employees, the exposure at site boundaries, and the load factor for all actions within a specific laboratory.

\subsubsection{Dose limits for employees}
See table on slide 7. Must be memorized.

\subsubsection{Load factor for rooms}
The classification of actions is based on the risk of internal contamination:
\begin{itemize}
	\item Chance of spreading $\rightarrow$ Dispersion parameter $p$
	\item Type of laboratory $\rightarrow$ Laboratory parameter $q$
	\item Type of ventilation $\rightarrow$ Ventilation parameter $r$
\end{itemize}
See table on slide 9.
\[ A_{max} = \frac{0.02 \cdot 10^{p+q+r}}{}\]

\subsection{Site boundary dose limit}
The cumulative dose at the site boundary should be < 40 mSv/y. At Randwyck, there are 4 points where it can be checked. All of the radiation is contained within 3 buildings (UNS50, MUMC+, and Maastro). There are exclaves in Venlo.

\subsection{Important points}
\begin{itemize}
	\item Be clear in the steps that the researcher needs to take
	\item Decide which steps are critical (based on exposure)
	\item Define which employees are helping in some steps
	\item Think about reducing exposure: can something be changed (time, distance, shielding)?
	\item Can different equipment be used?
	\item Can the experiment be performed in an alternative way?
	\item Always choose the most conservative point.
\end{itemize}

\section{Interactive case study: animal research with dispersible radioactive substances}
\subsection{Part 1}
What information is missing?
\begin{itemize}
	\item Is the fume hood formally tested? $\rightarrow$ Assume fumehood not formally tested.
\end{itemize}
What information is unnecessary?
What information is important?
\begin{itemize}
	\item 10 mice, 8 scans per mice (twice per week) = Total of 80 scans
	\item Each mouse gets 5 MBq per scan = Total weekly 100 MBq = Total of the whole study 400 MBq
	\item Everything takes places in a B-lab.
\end{itemize}