\section{Effects of radiation on cells}
Everything in the cell is a possible target. However, the biggest risk is the nucleus, and its DNA. This is because it can create DNA damage, such as base damage, cross-links, single/double strand breaks. In oncology, double strand breaks are needed to treat cancer. 1Gy of LET X-rays produces 1000 single-strand breaks, 40 double-strand breaks and 1000 altered bases.

\subsection{DNA repair}
Cells have a lot of repair mechanisms. Double-stranded breaks can cause many errors due to non-homologous end-joining. NHEJ cuts corners for the sake of speed. Single-strand breaks are easily repaired, in a couple of minutes after irradiation, 80\% of breaks are repaired. Double-strand breaks take much longer, taking hours for the same 80\% of breaks being repaired. 

\subsection{Stochastic effects}
Chance is proportionally related to the dose. There's no threshold, any exposure is any damage. It can cause cancer or congenital defects. There may be a latency period. This is based on the multiple hit theory (multiple 'hits' are needed to cause cancer). The multiple hit theory says that multiple genetic changes are necessary. Oncogenes (such as Ras) are needed to quickly proliferate cells. Tumour supressor genes (such as p53 or RB1) are needed to prevent mutated cells from dividing. DNA-repair genes (such as HNPCC or BRCA) are needed to to repair the DNA. Leukemia is a more immediate form of cancer (around 10-20 years), but most other forms of cancer take at least 15 years. \\
However, electromagnetic radiation usually only makes around 5-7\% of the carcinogenic factors. This also includes the sun's radiation, radon gas from working inside (radon is formed in the same rocks that are used to make buildings). Cancer is a multi-factorial risk, so it's difficult to tell the cause of the cancer. However, risks can be estimated by having large groups of exposed and non-exposed individuals, knowing the exact dose everyone received, and having a big difference with background radiation. One example of research was the one done after Hiroshima and Nagasaki. They had over 85k subjects in the cohort, and it has been running since the atomic bombs. The subjects get questionnaires, medical checkups, etc. yearly. Using the position where everyone was in the city when the bombs hit, the dose was estimated, which ranged from 0.01 to 6 Gy. By this life span study, they found that stomach cancer had an excess risk. There was also an increased chance of lung cancer, due to inhaling fallout. In general, per whole-body 1Gy dose, cancer risk increased by 47\%.  For those who received over 2 Gy, 56\% of cancers were caused by that dose. The age also increases the effect of the cancer. The younger at which you're exposed, the higher the chance of cancer. \\
The risk of malignancy was set to 5\%/Sv for 'civilians', and for workers 4\%/Sv. This is because in 'civilians' it's usually because of accidents, with a high dose at once; while in workers it's a lower dose spread through many years, which is more repairable. \\
Cancer is not the only type of stochastic effect. There's also genetic effects, which are hereditary. This damage is induced before conception, and may skip several generations. It is therefore difficult to study these effects in humans, so we only have animal studies. The studies also only use high doses (>0.5 Gy). A linear dose-effect relation without threshold is assumed. Only the damage in live births is accounted for. One such research was the MegaMouse project (>7000 mice). They looked at 6 types of hair colour and stunted ears. They found that not all the mice were equally sensitive, that repair time before procreation decreased the mutation frequency, and that there was a dose rate effect (high dose rate causes more damage than a low dose rate). From those experiments, there was a risk number of 1\%/Sv and a high spontaneous incidence of 10-20\% of genetic effects. The individual genetic effects are low. \\
Radiation during pregnancy is quite risky. The later the fetus is, the lower the risk to it is. During the preimplantation period (0-10 d), cells are pluripotent, so they can replace each other. Any damage results in apoptosis. It's an all-or-nothing effect, if the fetus survives, the child will not have effects. If not, the period will occur normally. During the organogenesis period (3-8 w), congenital malformations may occur. The effects are deterministic, with a threshold dose of 100 mSv. During the fetal period (8-25w) there's a growth delay, which may cause mantal retardation with a threshold dose of 100mSv, and after that a 10-40\% chance per Sv, with about 30 IQ points loss per Sv. The childhood malignancy is at 6\%/Sv. There's an increased risk of adult malignancy in life (2-3x). The fetal sensitivity is highest during the 1st trimester. With a dose below 100 mSv, the stochastic effects are unlikely, and there's no harmful tissue reactions. A dose above 500 mSv justifies abortion, as the risks are too high. The law sets an absolute limit from notification until birth of 1 mSv.

\subsection{Deterministic effects}
Deterministic/harmful tissue reactions depend on the dose. It takes a certain dose (threshold dose) to see effects in a certain amount of people (5\% of people see the effect). Below the threshold dose, the effects are unlikely. However, the sensitivity depends on the person, based on how fast their cells repair DNA. It seems like this resistance does not always transmit to the cancer.\\
With stochastic effects, even with a very low dose, there is a possibility of cancer, even if very small. It is a very cautious approach. That's why the dose given to the general population should be as low as reasonably achievable. The higher the cumulative lifetime dose, the higher the chance of getting effects (cancer or hereditary effects). However, it is a binary effect: you get cancer or you don't, regardless of dose.\\
DNA damage can cause cell death (resulting in the loss of organ function or sterility - harmful tissue reactions/deterministic effects), mutations (resulting in cancer or hereditary defects - stochastic effects), or repair. \\
Acute harmful tissue reactions can be seen immediately or delayed. Immediate reactions  are mainly cell loss, followed by an inflammatory response. Cell loss can result in anaemia, neutropenia, thrombopenia, epidermolysis, hair loss, ulceration... The following inflammatory response can result in mucositis, cystitis, enteritis, encephalitis, erythema, periostitis, keratitis... Late harmful tissue reactions can include atrophy, damage to blood vessels, chronic inflammatory reactions, fibrosis, sclerosis, necrosis... The effects vary by organ, as they have different sensitivities.\\
One famous case of carelessness is that of two interventional radiologists, and two nurses in 2 hospitals in Spain. They all developed cataract in both eyes, within two years. They were completely unknowledgeable about radiation protection. They didn't wear enough protection. There was no overhead shielding. They reached 0.45-0.9 Sv/y for several years.\\

\subsection{Dose limits}
In normal work, no harmful tissue reaction will be seen. There's dose limits which are set well below the damage threshold. The responsability of protection falls on the institution, not the worker. For 'civilians', the limit is 1 mSv/y. For radiation workers, the limit is 20 mSv/y. The damage is first seen at 2000 mSv.\\
%-------------------------------------------------------------------------------------------------------------------------------------------------------------------------------------------------------------------------------------------------------------------------------------------------------------------------------------%