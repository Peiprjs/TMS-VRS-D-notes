\section{Ways of radiation exposure}
Average yearly dose per person in the Netherlands
\begin{enumerate}
	\item Medical applications (1-1.2 mSv) - Medical diagnostics (X-ray, CT, PET, SPECT)
	\item Radon in housing (0.64-1.37 mSv)
	\item Food (0.43 mSv) - Vegetables, meat and fruit have $\ch{^{40}_{}K}$, $\ch{^{210}_{}Pb}$, and $\ch{^{210}_{}Po}$. Fish have $\ch{^{137}_{}Cs}$
	\item Building materials (0.34 mSv) - Concrete and sheet rock have $\ch{^{226}_{}Ra}$, $\ch{^{232}_{}Th}$, and $\ch{^{40}_{}K}$.
	\item Cosmic radiation (0.22 mSv) - Mostly charged particles ($P^+$ and $e^-$)
	\item Terrestial radiation (0.03 mSv) - Depends on the soil type (Granite has $\ch{^{238}_{}U}$)
	\item Air traffic (increased cosmic radiation) (0.04 mSv)
	\item Radiation from atomic bombings (<0.01 mSv)
\end{enumerate}
Total 2.8 mSv per year. Zuid-Limburg has a lot more background radiation than most of the Netherlands, due to higher amount of radon in residential houses. Mountain ranges tend to have higher background radiation. However, as far as science knows, radiation does not seem to have an effect in life expectancy. \\
Belgium has a higher amount of Radon because it's on top of a lot of granite deposits. In Belgium they also do a lot more CT than the rest of the world. In the US, if you can afford it, they give you a lot of CTs.\\

\section{Risks and effects of ionizing radiation}
Ionizing radiation is harmful for the exposed individual, as well as for the individual's offspring. It causes short and long-term effects. We know it's dangerous because of history (Radium girls, Radithor, X-ray shoe fittings, radioactive toothpaste...). However it does have a few benefits, from medical diagnostics to safety. 
\subsection{Effects}
After Hiroshima and Nagasaki, Leukemia peaked around 10 years later, and all-type cancer around 35 years later. There's a dose-effect relation. We work with low amounts of radiation, through a long time of exposure. The Life Span study has drawbacks: it was a high dose rate at a short time of exposure, it was a total body exposure, and it was a specific type of exposure. This makes it difficult to use to calculate risks associated to work. Per Sievert of total body exposure, there's a 4-5\% of developing a fatal cancer. Per Sievert of exposure to the gonads, there's a 1\% chance at developing severe genetic damage in the offspring.\\
Hormesis is the beneficial effects due to low levels of radiation (homeopathic). There's no scientific proof (yet). The accepted model is the linear-no threshold model (any radiation is harmful).\\

\section{Risk perception}
Public perception is quite negative, and it is mostly affected by a few accidents. Radiophobia is an unfounded perceived risk.\\
MUMC+, UM and Maastro produce a lot of radioactive waste. Risk perception about radiation is much higher than it should be.\\
\begin{table}[]
\begin{tabular}{ll}
Expert                       & Layman                     \\
Based on evidence            & Based on emotion           \\
Nuanced decision             & Binary decision            \\
Weighing aspects             & Binary decision            \\
Relative risk                & Specific events            \\
Averaged over the population & Personal consequences      \\
High level of understanding  & Low level of understanding
\end{tabular}
\end{table}
