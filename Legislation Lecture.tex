\section{Formation of legislation}
The ICRP makes recommendations. The Euratom turns them into guidelines, which are the basic safety standards. EU guidelines get made from that, which is the obligatory legislation for each member state. Finally, members tweak those rules and set higher standards. This process can take 18 years.

\section{Radiation protection}
The ICRP issues three main principles:
\begin{enumerate}
	\item Justificate why radiation is used, weighing advantages and disadvantages
	\item Limit the risk at chance related effects to acceptable levels	
	\item Prevent the occurrence of tissue reactions
\end{enumerate}
This translates into
\begin{enumerate}
	\item Justification
	\item ALARA (As Low As Reasonably Achievable)
	 \item Dose limits
\end{enumerate}

\section{Dutch Law}
Licenses are needed, which are issued by the ANVS. There are three types of licenses in the Netherlands:
\begin{enumerate}
	\item Single license (1-10 sources or devices), such as industry or dentists.
	\item Collection license (>10 sources or devices), such as small medical centers.
	\item Complex license (many complex and diverse actions, many sources or devices). UM, MUMC+, Maastro Clinic, Maastro Proton Therapy BV, and Brightlands Incubators Maastricht BV have one complex license.
\end{enumerate}

\subsection{Complex License Randwyck}
A Radiation Protection Unit is obligatory. This unit manages the license, issues internal permits, and acts as a supervisor on behalf of the entrepeneur. The General Coordinating Expert is mandated by the boards of all institutions. \\
The complex license randwyck is licensed to 5 locations, with a maximum of 115 X-ray devices, 5 linear accelerators, 1 cyclotron, 35 laboratories, 20 GBq of sealed sources... a maximum of 200g of fissionable materials, a maximum of 600 $Re_{inh}$ (Radiotoxicity equivalent, inhaled), storage of solid and liquid (25 kL) radioactive waste. \\

\subsection{Inspection}
Compliance is enforced by inspections by different departments.

\subsection{Protection of employees and environment}
\subsubsection{Classification of employees}
\begin{itemize}
	\item Category A employees - Interventionists (60). Have active monitoring.
	\item Category B employees - Mainly researchers (400). Have active monitoring.
	\item Category C employees - Such as transport employees, and some researchers (700). No active monitoring, but have risk exposure.
	\item Members of the public - Includes employees that aren't exposed at all to radiation.
	\item Pregnant employees - Dose limit of 1 mSV to the abdoment from the moment of announcing the pregnancy until birth. Regular tasks within the allowed exposure can be continued, though other tasks may be assigned after risk analysis. There are no reduced dose limits when trying to become/get someone pregnant.
\end{itemize}

\subsubsection{Dose limits}
See table in slide 15. The eye lens is especially sensitive to radiation, and with a few mSv, cataracts can be developed.\\
\textbf{In case of radiological emergencies:} out of free will, employees can receive up to per emergency (but it doesn't count towards the yearly limit):
\begin{itemize}
	\item Employees that act as public safety officers - 100 mSv
	\item Employees saving important materiaitemizeic interests - 250 mSv
	\item Employees saving lives - 500 mSv
\end{itemize}

\subsubsection{Dose restrictions}
There's the legal obligation to implement dose restrictions. Those are the target value for the maximum dose for an employee. Those must be reaitemizeic, and based on risk analysis. These are lower than de dose limit, which differs for each internal permit, and can be adjusted when necessary. Those aren't hard targets, but rather soft internal goals.

\subsubsection{Dosimetry}
Monitoring the exposure of individual employees is a legal obligation. Specific dosimetry is needed for specific jobs. There are multiple types of dosimetry devices.
\begin{itemize}
	\item Photon TLD badge
	\item Neuron TLD badge
	\item Photon/Beta TLD badge
	\item Ring TLD badge
	\item Electronic Personal Dosimeter (EPD)
	\item OSL badge (similar to PTLD, but improved)
\end{itemize}
A personal dosimeter is exchanged every month, results are available 1-3 months later. It must always be worn when working (not needed in the office), worn at chest height, and on top of the lead apron. It shouldn't be taken on airplanes, and the must be handed in on time (and not lost). The risk analysis is leading.
\begin{itemize}
	\item Category C, no personal dose monitoring needed
	\item Category B employees wear a personal dose monitor, depending on the type of work. A yearly (and start and end) mandatory routine questionnaire is done. Eye and blood tests are optional.
	\item Category A employees wear a personal dose monitor, depending on the type of work. A yearly (and start and end) mandatory routine questionnaire, eye and blood tests are done. 
\end{itemize}

\subsubsection{Storage facility for dispersible radioactive substances}
\begin{itemize}
	\item Dose rate may not exceed 1 $\mu$Sv/h at 10cm,
	\item Fire resistance of at least 60 minimum (fire should not go in)
	\item Access is restricted to authorized personnel
\end{itemize}

\subsubsection{Classification of work areas/zones}
\begin{itemize}
	\item Supervised zones (1 mSv/y < zone < 6 mSv/y - Categories A, B and C)
	\item Controlled zones (6 mSv/y < zone < 20 mSv/y - Category A only)
\end{itemize}
Work areas must be marked with safety signs and symbols. In case of emergency, the risks will be indicated before entering the room. If the dose rate is higher than 10$\mu$Sv/h an extra sign must be displayed.\\
Room classification comes with requirements:
\begin{itemize}
	\item Ventilation
	\item Equipment, such as a fume hood or detectors
	\item Finishing of materials (easy to clean)
	\item Organizational measures
	\item Changing rooms and specific clothing
	\item Room pressure must be below ambient pressure
	\item Fire safety
\end{itemize}

\subsubsection{Maximum permissible activity}
There's a maximum permissible activity in radionucleide laboratories. This can be calculated using a mathematical method, based on the risk of inhalation. Check pages 192-193 of the book.\\
The p factor is the chance at dispersion, which dictates the chance at inhalation/exposure. \\
The q factor is the lab classification, a parameter assigned to a type of laboratory.\\
The r factor is the ventilation factor, which depends on where the radioactive substance is being manipulated.\\
The load factor for working areas must be below 1. If the load factor approaches 1, you must change rooms. In the RNL there's multiple labs with multiple different purposes.

\subsubsection{Transport regulations}
All vehicles must comply with the same regulations. For external transport, they must have proper packaging, labelling and shielding. Only certified couriers are allowed to bring it, and they must follow ADR-7. For internal transport, only proper packaging, labelling and shielding is needed, but public roads cannot be crossed, so they are bypassed by using tunnels or bridges. A label is needed depending on the transport index TI and activity. It's based on the dose rate. \\
Internal transport must have proper packaging, labelling and shielding. The dose rate must be as low as possible. There's set routes through the buildings. There must always be permission from the sender and the receiver. Public roads must not be crossed. A trolley or other transportation device must be used. Elevators are a point of conflict.

\subsubsection{Occupational exposure}
May be external irradiation or internal contamination (inhalation of gases or aerosols, ingestion, sharps injury or wound contamination -from dispersible sources-). For external irradiation time, distance, and shielding must be considered. \textbf{ALARA} must always be applied.\\
The legal maximum allowed contamination is 0.4 Bq/cm² for $\alpha$-emitters, or 4 Bq/cm² for $\beta/\gamma$-emmiters.


